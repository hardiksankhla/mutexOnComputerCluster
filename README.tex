\documentclass[12pt]{article}
\usepackage{makeidx}
\usepackage{multirow}
\usepackage{multicol}
\usepackage[dvipsnames,svgnames,table]{xcolor}
\usepackage[dvips]{graphicx}
\usepackage{ulem}
\usepackage{hyperref}
\usepackage{amsmath}
\usepackage{amssymb}
\author{harsh}
\title{}
\usepackage[paperwidth=612pt,paperheight=792pt,top=56pt,right=56pt,bottom=56pt,left=56pt]{geometry}

\makeatletter
	\newenvironment{indentation}[3]%
	{\par\setlength{\parindent}{#3}
	\setlength{\leftmargin}{#1}       \setlength{\rightmargin}{#2}%
	\advance\linewidth -\leftmargin       \advance\linewidth -\rightmargin%
	\advance\@totalleftmargin\leftmargin  \@setpar{{\@@par}}%
	\parshape 1\@totalleftmargin \linewidth\ignorespaces}{\par}%
\makeatother 

% new LaTeX commands
\newcommand{\styleBullets}[1]{#1}
\newcommand{\styleHeading}[1]{{\large #1}}
\newcommand{\styleIndex}[1]{#1}
\newcommand{\styleTextbody}[1]{#1}


\begin{document}

\begin{enumerate}
	\item This zip folder contains a file named
\hspace{15pt}Y10UC172\_Y10UC122\_Y10UC125.pdf that tells us about the
basic definition of our project and all the references that we have
used. It also contains information about the various algorithms of
mutex that we have used in our computer cluster.
	\item File named randa.c contains the c code containing the header file
mpi.h that works on the ricarts and agarwala's algorithm.
	\item File named cluster.pdf contains the commands which we have used in all
the nodes to configure and make the cluster.
	\item File named algorithm.pdf contains the Ricarts and Agarwala's
algorithm.
	\item To compile the MPI code, we have to type the following command:
\end{enumerate}


{\raggedright
mpiCC \textendash{}o randa.c randa
}

{\raggedright
To run the MPI code we have to type the following command:
}

{\raggedright
\hspace{15pt}mpirun \textendash{}np 3 randa.c
}


\end{document}